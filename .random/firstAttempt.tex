\documentclass[crop=false]{standalone}


\usepackage{amsmath}  % align enviroment
\usepackage{ifthen}       % Guess what it does, I dare you
\usepackage{xparse}     % Usefull command creation macros


\ProvideDocumentCommand{\ode}{ s O{t} m O{} }{%
    \IfBooleanTF {#1}{%
        \frac{\mathrm{d}^{#4}}{\mathrm{d}\displaystyle#2^{#4}}\displaystyle#3
        }{%
        \frac{\mathrm{d}^{#4}\displaystyle#3}{\mathrm{d}\displaystyle#2^{#4}}
        }
    }
\ProvideDocumentCommand{\pde}{ s m    m O{} }{%
    \IfBooleanTF {#1}{%
        \frac{\partial^{#4}}{\partial\displaystyle#2^{#4}}\displaystyle#3
        }{%
        \frac{\partial^{#4}\displaystyle#3}{\partial\displaystyle#2^{#4}}
        }
    }


\begin{document}


\begin{align}
    \ode{V} && \ode*{V} && \ode[x]{V} && \ode{V}[2] && \ode[x]{V}[3] && \ode*[c_0]{V}[69] \\
    \nonumber \\
    \pde{t}{V} && \pde*{t}{V} && \pde{x}{V} && \pde{t}{V}[2] && \pde{x}{V}[3] && \pde*{c_0}{V}[69] 
\end{align}


\end{document}


% \def\Com{\futurelet\testchar\MaybeOptArgCom}
% \def\MaybeOptArgCom{\ifx[\testchar%
%                     \let\next\OptArgCom%
%                     \else%
%                     \let\next\NoOptArgCom%
%                     \fi \next}

% \def\aa{\futurelet\next\aaX}
% \def\aaX{%
% \ifx[\next \expandafter\aaXX
% \else \expandafter\aaXXX \fi}

% \def\aaXX[#1]#2{%  
% Opt: [#1] Req: [#2]\par}

% \def\aaXXX#1{%    
% Opt: [#1] Req: [#1]\par}

% \def\aaXXX#1{\aaXX[#1]{#1}}
% \def\aaXXX#1{\aaXX[<default>]{#1}}

% The trick here is to scoop up the first argument and
% store it away:
% \def\bb#1{\def\savedargone{{#1}}%
%     \futurelet\next\bbX}
    
% After that, we proceed for a while as before
% \def\bbX{%
%     \ifx[\next \expandafter\bbXX
%     \else \expandafter\bbXXX \fi}
% \def\bbXXX#1{\bbXX[#1]{#1}}

% and then we insert the saved argument in between
% the final macro and the the remaining arguments:
% \def\bbXX{\expandafter\bbY\savedargone}
%     \def\bbY#1[#2]#3{%  
%     One: [#1] Opt: [#2] Req: [#3]\par}
    
% The \expandafter in \bbXX turns the sequence

% \bbXX <arg 2><arg 3>

% which first becomes

% \expandafter\bbY\savedargone
% <arg 2><arg 3>

% into

% \bb <arg1><arg 2><arg 3>
