\PassOptionsToPackage{hidelinks, breaklinks= true,
linkcolor = [rgb]{0,0,0}, urlcolor  = [rgb]{0,0,0}, citecolor = [rgb]{0,0,0},
pdfdisplaydoctitle = true,
pdfkeywords={LaTeX, dtx, source, odesandpdes, ODE, PDE, differentials},
pdfsubject={Optimizing useage of ODE and PDE commands for LaTeX},
pdfauthor={Anakin}, pdftitle={The odesandpdes package}}{hyperref}
\documentclass[11pt,a4paper]{ltxdoc}
\usepackage[T1]{fontenc}
\usepackage{indentfirst}
\usepackage[centering, vscale = 0.80, hscale = 0.65]{geometry}
\usepackage{mathptmx,amsmath,fdsymbol}
\usepackage{odesandpdes}
\usepackage{tikz}
\usetikzlibrary{graphs,quotes}
\makeatletter
\setlength{\parskip}{5\p@ plus2\p@ minus2\p@}
\setlength{\jot}{7\p@}
\makeatother
\begin{document}
\author{Anakin\\ \texttt{anakin@ruc.dk}}
\title{The \textsf{odesandpdes} package}

\maketitle

\begin{abstract}
This package is the solution no one asked for, to a problem 
nobody had. Have you ever thought to yourself ``wow, I sure do
dislike having to remember \emph{multiple} macros for my odes and pdes''
and the author of this package has to agree, wholeheartedly.
In the modern world of ``tik-toking'' and ``family guy surfing'', 
our brains have rotted beyond salvage for even basic levels of 
cognitive recall. This package aims to fix this, through two
macros that have been set to each have an identical form and 
function, with an emphasis on intuitive use.
Through setting options, the multiple
common notational style are easily
swapped between, all by a single option.
\emph{You're welcome}.
\end{abstract}




{\setlength{\parskip}{0.25ex}\small
\tableofcontents}



\newpage
\section*{My funny little ODE/PDE package}
\hspace{1em} Start by first having \verb|odesandpdes.sty| downloaded in an
accessible directory, or in the same directory as your 
overleaf main.tex, using it by inserting; 
\begin{center}
\cs{usepackage\oarg{options}\{odesandpdes\}}
\end{center}
into the preamble, Ideally after any font changing packages you use.

\section{Usage}

If the reader does not wish to be gradually introduced to the package
and its features, feel free to skip directly to section \ref{sec:examples}.

\subsection{Options}

\DescribeMacro{notation}
\DescribeMacro{maxprimes}
The options included are based off of the three most
common notations
(according to Wikipedia), Lagrange, Leibniz, and Newton. 
They can be accessed through the \oarg{options} when importing the package;
\par\hbox to \textwidth{\hss
\cs{usepackage[notation=\meta{option}]\{odesandpdes\}} \hss}

In the case of Lagrange or Newton notation, there is the |maxprimes| option
for determination of how many physical markings are allowed to be
made before the notation switches to a symbolic version;
\par\hbox to \textwidth{\hss
\cs{usepackage[maxprimes=\meta{integer}]\{odesandpdes\}}\hss}
\vspace{1ex}


\DescribeMacro{\setDE}
However, if one might wish to change it on a section to section basis,
the command \cs{setDE}\marg{options} is able
to take any package option as an argument and will
apply the new option going forward.


\par\hbox to \textwidth{\hss
\begin{tabular}{lcl}\hline
  Option list & Default Value & Valid Arguments \\ \hline
  notation & Leibniz & {default, Lagrange, Leibniz, Newton} \\
  maxprimes & 3 & $\text{maxprimes} = n, n \in \mathbb N_+$ \\ \hline
\end{tabular}\hss}



\subsection{The Meat and Potatoes}

\hspace{1em} The command(s) are approached with the philosophy
of of an intuitive and modular usage. 
The full extent of its usage can look like;
\begin{equation*} |\ode*[x]^2 X(x) =\ode T_{\eta} at 0; -\alpha| 
\Rightarrow
\ode*[x]^2 X(x)=\ode T_{\eta} at 0; -\alpha
\end{equation*}
very quickly, and very easily building complex interactions
of differentials.
The quick functional break down of each element that comprises the macro;
\newline
\centerline{ \cs{ode}\meta{star}\oarg{variable}\string^\meta{degree}
\marg{function}at\textvisiblespace\meta{position};}\vspace{1ex}
\par\hbox to \textwidth{\hss
\begin{tabular}{cl}\hline
   Argument &  Usage \\ \hline
   \oarg{variable} & The variable being derived \\ 
   \meta{degree} & The order/degree of the derivative \\
   \marg{function} & The function being derived \\
   \textvisiblespace at\textvisiblespace\meta{point}; 
      & Where the function is being derived \\ \hline
\end{tabular}\hss}\vspace{1ex}
All arguments are conditionally optional, only the function is
mandatory, but the command can forgo needing a function if a star is placed.

\subsubsection*{Notation Style}

\DescribeMacro{\LagrODE}
\DescribeMacro{\LeibODE}
\DescribeMacro{\NewtODE}
\DescribeMacro{\LagrPDE}
\DescribeMacro{\LeibPDE}
\DescribeMacro{\NewtPDE}
There are 3 distinct notational styles
one can choose between. This choice can be made as a package option
in the preamble, in the text with \cs{setDE}\marg{options}, or if 
one only needs to use a notation style once, through its respective 
macro.

In essense, all the \cs{ode} or \cs{pde} commands do are call the 
respective notational varient aligned with the currently set option.
This makes it simple enough to just use one of the notational varients,
should one wish to do so:
\begin{equation*} |\LagrODE[x] c = \LeibODE[x] c = \NewtODE[x] c |\quad
\Rightarrow\quad \LagrODE[x] c = \LeibODE[x] c = \NewtODE[x] c 
\end{equation*}
This also means that all these functions are identical in what arguments
they take.


\subsubsection*{Variable and Function Arguments}
\DescribeMacro{\ode}
\DescribeMacro{\ode*}
The most barebone form can be understood as:\par\noindent
\hbox to \textwidth{\hss\vbox{
\hbox{\cs{ode}\oarg{variable}\marg{ function}}
\hbox{\cs{ode*}\oarg{variable}}}\hss}

\DescribeMacro{\pde}
\DescribeMacro{\pde*}
and for the sake of parity, the PDE usage is identical:\par\noindent
\hbox to \textwidth{\hss\vbox{
\hbox{\cs{pde}\oarg{variable}\marg{ function}}
\hbox{\cs{pde*}\oarg{variable}}}\hss}\par
Any value you give to the \emph{optional} \oarg{variable} argument
will be represented as the variable being derived. 
While the \emph{mandatory} \marg{function} argument will be the function you
are deriving. 
Say you wish to indicate you are deriving $X(t)$, simple as writing 
|\ode[t]{X}|, however, its worth noting that $t$ is the default variable
so writing |\ode{X}| will produce identical results.
Hence |\ode[t]{X} = \ode{X}| will produce;
\begin{equation*} |\ode[t]{X} =\ode{X}| \implies \ode[t]{X} = \ode{X} 
\end{equation*}
 

While the \marg{function} argument is mandatory using the 
non-starred command, using the starred varient
omits the need for the \marg{function} argument.
Therefor, writing the exact same equation, just starred
|\ode*[t]{X} = \ode*{X}| will instead produce;
\begin{equation*} |\ode*[t]{X} =\ode*{X}| \implies \ode*[t]{X} = \ode*{X} 
\end{equation*}
Effectively one can rewrite the `bare-bones' display as:\par\vspace{1ex}
\par\hbox to \textwidth{\hss
\cs{ode}\meta{star}\oarg{variable}\marg{ function}
\hss}

\subsubsection*{Degree of Derivative}
The previously shown stated section is something the reader has
likely encountered before, made themselves. This is where
this package begins to differentiate\footnote{Calculus Pun!} itself.
Consider:
\par\hbox to \textwidth{\hss
\cs{ode}\meta{star}\oarg{variable}$\uparrow$\meta{degree}\marg{function}
\hss}

A feature of this family of commands, is that it can `\emph{easily}'
recognize a following exponent should one be placed. 
There was rational in choosing to check for the exponent immediately
after the macro command opposed to checking for the exponent at
the end after the function. 
As, often you would want add a higher degree very
quickly as opposed to \emph{after} defining the function. 

\hbox to \textwidth{\hss
\cs{ode}|^2{f(x)}| as opposed to \cs{ode\{f(x)\}}|^2| \hss}


This was one of the main motivations of creating a package to begin with
as instead of needing, maybe, two personalized commands,
such as ``|\ddt{f}| and |\ddxx{f}|'', or ``|\dd{x}{f}| and |\dd[2]{x}{f}|''.
One simply needs to treat the \cs{ode} macro itself as being raised
to a higher degree.
\begin{equation*} |\ode* \left(\ode{f} \right)=\ode^2{f} | 
\Rightarrow \ode* \left(\ode{f} \right)=\ode^2{f} 
\end{equation*}


\subsubsection*{Defining Where the Derivative is}

Imagine you, as the reader, are trying to quickly and easily
write up the boundry conditions of your problem.
One could always make another macro, in what is no doubt an impressive
display of differential shortcuts.
\emph{Or}: \vspace{1ex}
\par\hbox to \textwidth{\hss
\cs{ode}\meta{star}\oarg{variable}$\uparrow$\meta{degree}\marg{
function}\textvisiblespace{}at\textvisiblespace{}\meta{postion};
\hss}

See, \TeX\ does something very interesting when it uses `\emph{glue}',
which is partially replicated by packages such as |TikZ|, where it will
happily take `soft' modifiers written directly in plain english.
If one wishes to strictly define paragraph spacing in \TeX, they would use
`\cs{parskip}|=1ex|'. If one would rather give it a range of tolerance
the following construct `\cs{parskip}|=1ex plus 0.5ex minus 0.5ex|'
then allows a spacing of $1\pm 0.5$ |ex|.

Glue is of course something special, but that does not mean 
that the author can not gain inspiration. Say one wishes
to define Neumann boundries;
\begin{equation*} |\ode[x]{c} at 0;=0\land\ode[x]{c} at L;=1|
\Rightarrow \ode[x]{c} at 0;=0\land\ode[x]{c} at L;=1
\end{equation*}
\begin{equation*} |\ode[x]{c} at 0 = L;=1|
\Rightarrow \ode[x]{c} at 0 = L;=1
\end{equation*}
Literally could not be easier.\footnote{My source is that I made it up}


Those reading til this point may have recalled that the first example
did not contain many braces.
This is because with the ``proper'' spacing, there is little 
need for the use of the braces, so as to help promote a more fluid, 
(and readable),
workflow without always needing to worry about the f|***|ing brace. 
Not that one can not use the brace for personal taste. 
In the following section, many examples of use will be illustrated
to show the range and versitility of the functions.

\noindent
\fbox{\parbox{\textwidth}{The most important thing to always remember.
\emph{Just because} the author
of this package has done as much as they can to `\emph{\rlap{idiot}\hbox{------}
user proof}' its functions
does not mean the user does not still need to be cautious. This is 
\LaTeX\ we are talking about. There are likely many
scenarios that the author did not think of, nor accidentally came across.}}

\newpage
\section{Examples of use}\label{sec:examples}

\stepcounter{subsection}

\addcontentsline{toc}{subsection}{\thesubsection\quad Common Use Examples}

To show the generality of use. The following examples all take identical form 
in the \TeX/\LaTeX\ itself. 
Additionally, in order to illustrate the functional boundries of the command with
respect to each of the notational styles. 
There is a variety of spacing and bracketing to help highlight these features,
and will be shown in the following |verbatim| enviroment; 


\begin{minipage}{0.98\textwidth}
\begin{verbatim}
\begin{align*}
\ode A(x)      && \ode[x]{B(x)} && \ode^1 C(x)     && \ode[x]^5 {D(x)} \\
\ode* {E(x)}   && \ode*[x] F(x) && \ode*^2 {G(x)}  && \ode*[x]^6H(x)   \\
\pde[t] I(x)   && \pde[x] {J(x)}&& \pde[t]^3K(x)   && \pde[x]^7 {L(x)} \\
\pde*[t] {M(x)}&& \pde*[x]N(x)  && \pde*[t]^4 O(x) && \pde*[x]^8 P(x)
\end{align*}
\end{verbatim}
\end{minipage}

\vbox{\centering
\hbox{\verb|\setDE{notation=Lagrange}| \emph{and/or} \verb|\usepackage[notation=Lagrange]{odesandpdes}|}
\fbox{\parbox{0.65\textwidth}{
\setDE{notation=Lagrange}
\begin{align*}
  \ode A(x)      && \ode[x]{B(x)} && \ode^1 C(x)     && \ode[x]^5 {D(x)} \\
  \ode* {E(x)}   && \ode*[x] F(x) && \ode*^2 {G(x)}  && \ode*[x]^6H(x)   \\
  \pde[t] I(x)   && \pde[x] {J(x)}&& \pde[t]^3K(x)   && \pde[x]^7 {L(x)} \\
  \pde*[t] {M(x)}&& \pde*[x]N(x)  && \pde*[t]^4 O(x) && \pde*[x]^8 P(x)
\end{align*}
}}}\vspace{1.25em}

\vbox{\centering
\hbox{\verb|\setDE{notation=Leibniz}| \emph{and/or} \verb|\usepackage[notation=Leibniz]{odesandpdes}|}
\fbox{\parbox{0.65\textwidth}{
\setDE{notation=Leibniz}
\begin{align*}
  \ode A(x)      && \ode[x]{B(x)} && \ode^1 C(x)     && \ode[x]^5 {D(x)} \\
  \ode* {E(x)}   && \ode*[x] F(x) && \ode*^2 {G(x)}  && \ode*[x]^6H(x)   \\
  \pde[t] I(x)   && \pde[x] {J(x)}&& \pde[t]^3K(x)   && \pde[x]^7 {L(x)} \\
  \pde*[t] {M(x)}&& \pde*[x]N(x)  && \pde*[t]^4 O(x) && \pde*[x]^8 P(x)
\end{align*}
}}}\vspace{1.25em}

\vbox{\centering
\hbox{\verb|\setDE{notation=Newton}| \emph{and/or} \verb|\usepackage[notation=Newton]{odesandpdes}|}
\fbox{\parbox{0.65\textwidth}{
\setDE{notation=Newton}
\begin{align*}
  \ode A(x)      && \ode[x]{B(x)} && \ode^1 C(x)     && \ode[x]^5 {D(x)} \\
  \ode* {E(x)}   && \ode*[x] F(x) && \ode*^2 {G(x)}  && \ode*[x]^6H(x)   \\
  \pde[t] I(x)   && \pde[x] {J(x)}&& \pde[t]^3K(x)   && \pde[x]^7 {L(x)} \\
  \pde*[t] {M(x)}&& \pde*[x]N(x)  && \pde*[t]^4 O(x) && \pde*[x]^8 P(x)
\end{align*}
}}}

\vbox{\centering
\hbox{\verb|\setDE{maxprimes=7}| \emph{and/or} \verb|\usepackage[maxprimes=7]{odesandpdes}|}
\fbox{\parbox{0.65\textwidth}{
\setDE{notation=Lagrange,maxprimes=7}
\begin{align*}
\ode^1 f &&\ode^2 f &&\ode^3 f &&\ode^4 f &&
\ode^5 f &&\ode^6 f &&\ode^7 f &&\ode^8 f &&\ode^9 f 
\end{align*}
\setDE{notation=Newton}
\vspace{-1.5em}
\begin{align*}
\ode^1 f &&\ode^2 f &&\ode^3 f &&\ode^4 f &&
\ode^5 f &&\ode^6 f &&\ode^7 f &&\ode^8 f &&\ode^9 f 
\end{align*}
}}}



\subsection{"at x;" Usage Examples}



\hspace{1em} Now, because the author is not an insane person, and went through the 
effort of learning how TEX deconstructs text into constitute 
registries and boxes, the way any sane person might. When using 
a non-starred version of a command, after the function is defined, you can
place an `|at|\textvisiblespace\meta{point}|;|', and the representation will 
shown according to notational convention.


\vbox{
\begin{center}
\begin{minipage}[c]{0.45\textwidth}
\begin{verbatim}
\begin{align*}
   \ode[x]  c at 23\pi;   &= 1 \\
   \ode[x]^3 c   at 69;   &= 2 \\
   \ode[x]^{69} c at L;+t &= 3 \\
   \ode[x]^9  c af 420;   &= 4 \\
   \ode[x]^6  c  a t 13;  &= 5 
\end{align*}
\end{verbatim}
\end{minipage}
\end{center}
\noindent
\hbox{\begin{minipage}[t]{0.35\textwidth}
\setDE{notation=Lagrange}
\noindent\setlength{\jot}{2em}
\begin{verbatim}
\setDE{notation=Lagrange}
\end{verbatim}
\vspace{-1em}
\begin{align*}
   \ode[x]  c at 23\pi;   &= 1 \\
   \ode[x]^3 c   at 69;   &= 2 \\
   \ode[x]^{69} c at L;+t &= 3 \\
   \ode[x]^9  c af 420;   &= 4 \\
   \ode[x]^6  c  a t 13;  &= 5 
\end{align*}
\end{minipage}}\vline~
\hbox{\begin{minipage}[t]{0.34\textwidth}
\setDE{notation=Leibniz} 
\noindent\setlength{\jot}{0.70em}
\begin{verbatim}
\setDE{notation=Leibniz}
\end{verbatim}
\vspace{-1em}
\begin{align*}
   \ode[x]  c at 23\pi;   &= 1 \\
   \ode[x]^3 c   at 69;   &= 2 \\
   \ode[x]^{69} c at L;+t &= 3 \\
   \ode[x]^9  c af 420;   &= 4 \\
   \ode[x]^6  c  a t 13;  &= 5 
\end{align*}
\vphantom{l}
\end{minipage}}\vline~
\hbox{\begin{minipage}[t]{0.32\textwidth}
\setDE{notation=Newton}
\noindent\setlength{\jot}{1.75em}
\begin{verbatim}
\setDE{notation=Newton}
\end{verbatim}
\vspace{-1em}
\begin{align*}
   \ode[x]  c at 23\pi;   &= 1 \\
   \ode[x]^3 c   at 69;   &= 2 \\
   \ode[x]^{69} c at L;+t &= 3 \\
   \ode[x]^9  c af 420;   &= 4 \\
   \ode[x]^6  c  a t 13;  &= 5 
\end{align*}
\end{minipage}}}

\hspace{1em} As can be seen in the examples, this `\emph{modifier}' is robust
enough that one can write effectively any combination of characters
after the function, excluding, \emph{verbatim}, `|at|\textvisiblespace' 
and it will work as intended.

\vbox{
\hspace{1em} \emph{Important to note}, due to a slight difference in how the
notational styles are defined, 
only the Leibniz notation can take arguments for the 
function that involve subscripts and superscripts without delimiters.
Mostly easily illustrated in this following 
example using the \cs{pde} command;
\begin{center}
\begin{minipage}[c]{0.45\textwidth}
\begin{verbatim}
\begin{align*}
  \pde[y]   f_1         &= 1 \\
  \pde[y]   f_1   at L; &= 2 \\
  \pde[y]   f     at L; &= 3 \\
  \pde[y] {(f_1)}       &= 4 \\
  \pde[y] {(f_1)} at L; &= 5
\end{align*}
\end{verbatim}
\end{minipage}
\end{center}
\noindent
\hbox{\begin{minipage}[t]{0.35\textwidth}
\setDE{notation=Lagrange}
\noindent\setlength{\jot}{2.20em}
\begin{verbatim}
\setDE{notation=Lagrange}
\end{verbatim}
\vspace{-1em}
\begin{align*}
  \pde[y]   f_1         &= 1 \\
  \pde[y]   f_1   at L; &= 2 \\
  \pde[y]   f     at L; &= 3 \\
  \pde[y] {(f_1)}       &= 4 \\
  \pde[y] {(f_1)} at L; &= 5
\end{align*}
\end{minipage}}\vline~
\hbox{\begin{minipage}[t]{0.34\textwidth}
\setDE{notation=Leibniz} 
\noindent\setlength{\jot}{0.70em}
\begin{verbatim}
\setDE{notation=Leibniz}
\end{verbatim}
\vspace{-1em}
\begin{align*}
  \pde[y]   f_1         &= 1 \\
  \pde[y]   f_1   at L; &= 2 \\
  \pde[y]   f     at L; &= 3 \\
  \pde[y] {(f_1)}       &= 4 \\
  \pde[y] {(f_1)} at L; &= 5
\end{align*}
\vphantom{l}
\end{minipage}}\vline~
\hbox{\begin{minipage}[t]{0.32\textwidth}
\setDE{notation=Newton}
\noindent\setlength{\jot}{2.20em}
\begin{verbatim}
\setDE{notation=Newton}
\end{verbatim}
\vspace{-1em}
\begin{align*}
  \pde[y]   f_1         &= 1 \\
  \pde[y]   f_1   at L; &= 2 \\
  \pde[y]   f     at L; &= 3 \\
  \pde[y] {(f_1)}       &= 4 \\
  \pde[y] {(f_1)} at L; &= 5
\end{align*}
\end{minipage}}
}

\subsection{Prime Count Limits}
\hspace{1em} Because the Newton and Lagrange notation is procedural;
the only limit is your imagination, and also the fact that 
\TeX\ can only have something like 127 unplaced tokens at a time.\par
\hbox to \textwidth{\hss\cs{setDE\{maxprimes=69\}}\hss}
\fbox{\parbox{\textwidth}{
\setDE{maxprimes=69}
\begin{minipage}{0.45\textwidth}
\setDE{notation=Lagrange}
\begin{equation*}
   \begin{split}
      \ode^{5}  f \\
      \ode^{16} f \\
      \ode^{32} f \\
      \ode^{54} f \\
      \ode^{69} f \\
      \ode^{70} f \\
   \end{split}
\end{equation*}
\end{minipage}~
\begin{minipage}{0.05\textwidth}
\setDE{notation=Lagrange}
\begin{equation*}
   \begin{split}
      \boxed{5}  \\
      \boxed{16} \\
      \boxed{32} \\
      \boxed{54} \\
      \boxed{69} \\
      \boxed{70} \\
   \end{split}
\end{equation*}
\end{minipage}~
\begin{minipage}{0.35\textwidth}
\setDE{notation=Newton}
\begin{equation*}
      \ode^{5}  f \quad
      \ode^{16} f \quad
      \ode^{32} f \quad
      \ode^{54} f \quad
      \ode^{69} f \quad
      \ode^{70} f \quad
\end{equation*}
\begin{equation*}
      \mkern-15mu\boxed{5}  
      \boxed{16}
      \boxed{32} 
      \boxed{54}
      \boxed{69} 
      \boxed{70} 
\end{equation*}
\end{minipage}}}


\end{document}